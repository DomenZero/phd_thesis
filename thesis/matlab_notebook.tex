%!TEX root = thesis.tex
%-------------------------------------------------------------------------------
\chapter{Software}
\label{chap:AppendixA}
%-------------------------------------------------------------------------------

%-------------------------------------------------------------------------------
\section*{Setup}
%-------------------------------------------------------------------------------
%-------------------------------------------------------------------------------
\subsection*{Requirements}
%-------------------------------------------------------------------------------

\begin{itemize}
    \item MATLAB,
    \item CVX  $\geq$ 2.1 \cite{Grant2008a},
    \item QETLAB $\geq$ 0.8 \cite{Johnston2015}.
\end{itemize}

%-------------------------------------------------------------------------------
\subsection*{List of functions}
%-------------------------------------------------------------------------------

\begin{itemize}
%    \item \texttt{HedgingValue} --- Semidefinite programs to determine the maximum and minimum probabilities with which Alice is able to hedge at least one game out of $r$ games played. 
    
    \item \texttt{MonogamyGameValueUB} (by N. Johnston) ---
        Given a monogamy-of-entanglement game, $G$, the function calculates an upper bound on the quantum value of $G$;
        
    \item \texttt{MonogamyGameValueLB} ---
        Given a monogamy-of-entanglement game, $G$, the function calculates an lower bound on the quantum value of $G$;
        
    \item \texttt{MUB} (by N. Johnston) --- generates a set of mutually unbiased bases for a given dimension;
\end{itemize}

\sloppy
\definecolor{lightgray}{gray}{0.5}
\setlength{\parindent}{0pt}

%-------------------------------------------------------------------------------
\section{Software Listings}
%-------------------------------------------------------------------------------

All of the following software listings in this Appendix are hosted on the Github repository found here~\cite{Russo2016c}. 

%\subsection{Semidefinite programs for computing the minimum probability that Alice loses all games in $G_H^n$ (Section~\ref{sec:hedging}) } \label{code:hedging-sdps}
%
%The following listing is hosted on the Github repository found here~\cite{Russo2015b}. 
%
%\begin{verbatim}
%function [ x ] = HedgingValue( P0, P1, n, k )
%
%[Q0,Q1] = CalculateQ( P0, P1, n, k );
%
%cvx_precision best
%
%% m(a) Primal
%cvx_begin sdp quiet
%    variable X(4^n, 4^n) semidefinite
%    minimize ( ip(Q0,X) )
%    subject to
%        PartialTrace(X,[1:2:2*n],2*ones(1,2*n)) == eye(2^n);
%cvx_end
%X_m = X; 
%
%% m(a) Dual
%cvx_begin sdp quiet
%    variable Y(2^n, 2^n) hermitian
%    maximize( trace(Y) )
%    subject to
%        PiPerm(n) * Tensor( eye(2^n), Y ) * PiPerm(n)' <= Q0; 
%cvx_end
%
%end
%\end{verbatim}
%
%\subsection{Parallel repetition of an interactive proof system with Bell-like inequality violation (Section~\ref{sec:hedging}) } \label{code:hedging}
%
%The following listing is hosted on the Github repository found here~\cite{Russo2015b}. 
%
%\begin{verbatim}
%% Define standard qubits as vectors
%e0 = [1;0]; e1 = [0;1]; 
%e00 = kron(e0,e0); e01 = kron(e0,e1); 
%
%% The state: \alpha|00> + sqrt(1 - \alpha^2)|11> where \alpha = 1/sqrt(2).
%u = 1/sqrt(2)*e00 + 1/sqrt(2)*e11; 
%rho = u*u'; 
%
%% The state: \cos(\theta)|00> + \sin(\theta)|11> where \theta = \pi/8.
%v = cos(pi/8)*e00 + sin(pi/8)*e11; 
%sigma = v*v'; 
%
%% Alice's measurement operators:
%P1 = v*v';							            % Winning measurement
%P0 = eye(4) - P1;				     % Losing measurement
%P1 = P1/2; P0 = P0/2; 	% Uniform prob. of selecting winning or losing measurement
%
%k = 1;		 % How many games Bob wins with certainty. 
%n = 2; 		% How many total games played. 
%
%% Probability of losing both games is 0.
%h_val = HedgingValue(P0,P1,n,k)
%\end{verbatim}\color{lightgray} 
%\begin{verbatim}     
%h_val =
%
%    -6.86624e-13
%\end{verbatim}
%\color{black}

%-------------------------------------------------------------------------------
\subsection{The first level of the extended QC hierarchy for the BB84 extended nonlocal game} \label{code:first-level-qc-hierarchy-bb84}
%-------------------------------------------------------------------------------

\begin{verbatim}
e0 = [1;0]; e1 = [0;1]; ep = [1;1]/sqrt(2); em = [1;-1]/sqrt(2);
psi0_dm = e0*e0'; psi0_dmc = e1*e1';
psi1_dm = ep*ep'; psi1_dmc = em*em';

R00 = psi0_dm/2;
R01 = psi0_dmc/2; 
R10 = psi1_dm/2;
R11 = psi1_dmc/2;  

dim = 9;
A00_B00 = zeros(dim); A01_B01 = zeros(dim); 
A10_B10 = zeros(dim); A11_B11 = zeros(dim); 

% These are the relative positions of these 
% entries as indexed by strings in the matrix. 
A00_B00(2,6) = 1; A00_B00(6,2) = 1;
A01_B01(3,7) = 1; A01_B01(7,3) = 1;
A10_B10(4,8) = 1; A10_B10(8,4) = 1;
A11_B11(5,9) = 1; A11_B11(9,5) = 1;

A = 1/2*( kron(R00, A00_B00) + kron(R01, A01_B01) ) + ...
	1/2*( kron(R10, A10_B10) + kron(R11,A11_B11) );

cvx_begin sdp
	cvx_precision best
    %#ok<*VUNUS>    % suppress MATLAB warnings for equality checks in CVX
    %#ok<*EQEFF>    % suppress MATLAB warnings for inequality checks in CVX 
    
    % Admissible matrix
    variable M(2*dim,2*dim) hermitian
    
    % Sub-block matrices found in the admissible matrix
    variable M11(dim,dim)
    variable M12(dim,dim)
    
    variable M21(dim,dim)
    variable M22(dim,dim)
            
    M == [ M11 M12;
    	   M21 M22 ];
       
    maximize trace( A*M )      
    
    subject to 
 
    % Normalization condition:
    M11(1,1) + M22(1,1) == 1;
        
    for i = 1:dim
        for j = 1:dim
            % Ensure commutation relation holds
            %(i.e. [A,B] = 0)            
            M11(i,j) == M11(j,i);
            M12(i,j) == M12(j,i);
            M21(i,j) == M21(j,i);
            M22(i,j) == M22(j,i);
            
            % Enforce operators as projective measurements
            % (i.e. the square of the same operator is found in the top
            % column / row of the diagonal entry). 
            M11(i,i) == M11(1,i);
            M11(i,i) == M11(i,1);
                        
            M12(i,i) == M12(1,i);
            M12(i,i) == M12(i,1);

            M21(i,i) == M21(1,i);
            M21(i,i) == M21(i,1);

            M22(i,i) == M22(1,i);
            M22(i,i) == M22(i,1);
        end
    end  

	% Enforce that projective measurements sum to 1:
    for i = 1:dim
    	for j = 1:dim
    		if mod(i,2) == 0    			
    		M11(i,j) + M11(i+1,j) == M11(1,j);
    		M12(i,j) + M12(i+1,j) == M12(1,j);
			M21(i,j) + M21(i+1,j) == M21(1,j);
			M22(i,j) + M22(i+1,j) == M22(1,j);
    		end
    		if mod(j,2) == 0
    		M11(i,j) + M11(i,j+1) == M11(i,1);
    		M12(i,j) + M12(i,j+1) == M12(i,1);
    		M21(i,j) + M21(i,j+1) == M21(i,1);
    		M22(i,j) + M22(i,j+1) == M22(i,1);   		    		
	    	end
    	end
    end

    % Ensure that the matrix is PSD.
    M >= 0;
cvx_end
\end{verbatim}
\color{black}
\color{lightgray} 
\begin{verbatim}     
cvx_optval =

    0.8536
\end{verbatim}
\color{black}

%-------------------------------------------------------------------------------
\subsection{The first level of the extended QC hierarchy for the CHSH extended nonlocal game} \label{code:first-level-qc-hierarchy-chsh}
%-------------------------------------------------------------------------------

\begin{verbatim}
e0 = [1;0]; e1 = [0;1]; ep = [1;1]/sqrt(2); em = [1;-1]/sqrt(2);
psi0_dm = e0*e0'; psi0_dmc = e1*e1';
psi1_dm = ep*ep'; psi1_dmc = em*em';

R00 = psi0_dm/2;
R01 = psi0_dmc/2; 
R10 = psi1_dm/2;
R11 = psi1_dmc/2;  

dim = 9;
A00_B00 = zeros(dim); A01_B01 = zeros(dim); 
A10_B10 = zeros(dim); A11_B11 = zeros(dim); 

% These are the relative positions of these entries as
% indexed by strings in the matrix. 
A00_B00(2,6) = 1; A00_B00(6,2) = 1;
A01_B01(3,7) = 1; A01_B01(7,3) = 1;
A10_B10(4,8) = 1; A10_B10(8,4) = 1;
A11_B11(5,9) = 1; A11_B11(9,5) = 1;

A00_B00 = zeros(dim); A01_B01 = zeros(dim); 

A00_B10 = zeros(dim); A01_B11 = zeros(dim);

A10_B00 = zeros(dim); A11_B01 = zeros(dim); 

A10_B11 = zeros(dim); A11_B10 = zeros(dim); 

A00_B00(2,6) = 1; A00_B00(6,2) = 1;
A00_B10(2,8) = 1; A00_B10(8,2) = 1;

A01_B01(3,7) = 1; A01_B01(7,3) = 1;
A01_B11(3,9) = 1; A01_B11(9,3) = 1;

A10_B00(4,6) = 1; A10_B00(6,4) = 1;
A10_B11(4,9) = 1; A10_B11(9,4) = 1;

A11_B01(5,7) = 1; A11_B01(7,5) = 1;
A11_B10(5,8) = 1; A11_B10(8,5) = 1;

% CHSH ENLG
A = 1/4*(kron(R00, A00_B00) + kron(R01, A01_B01)) + ...
    1/4*(kron(R00, A00_B10) + kron(R01, A01_B11)) + ...
    1/4*(kron(R00, A10_B00) + kron(R01, A11_B01)) + ...
    1/4*(kron(R10, A10_B11) + kron(R11, A11_B10));

cvx_begin sdp
	cvx_precision best
    %#ok<*VUNUS>    % suppress MATLAB warnings for equality checks in CVX
    %#ok<*EQEFF>    % suppress MATLAB warnings for inequality checks in CVX 
    
    % Admissible matrix
    variable M(2*dim,2*dim) hermitian
    
    % Sub-block matrices found in the admissible matrix
    variable M11(dim,dim)
    variable M12(dim,dim)
    
    variable M21(dim,dim)
    variable M22(dim,dim)
            
    M == [ M11 M12;
    	   M21 M22 ];
       
    maximize trace( A*M )      
    
    subject to 
 
    % Normalization condition:
    M11(1,1) + M22(1,1) == 1;
        
    for i = 1:dim
        for j = 1:dim
            % Ensure commutation relation holds
            %(i.e. [A,B] = 0)            
            M11(i,j) == M11(j,i);
            M12(i,j) == M12(j,i);
            M21(i,j) == M21(j,i);
            M22(i,j) == M22(j,i);
            
            % Enforce operators as projective measurements
            % (i.e. the square of the same operator is found in the top
            % column / row of the diagonal entry). 
            M11(i,i) == M11(1,i);
            M11(i,i) == M11(i,1);
                        
            M12(i,i) == M12(1,i);
            M12(i,i) == M12(i,1);

            M21(i,i) == M21(1,i);
            M21(i,i) == M21(i,1);

            M22(i,i) == M22(1,i);
            M22(i,i) == M22(i,1);
        end
    end  

	% Enforce that projective measurements sum to 1:
    for i = 1:dim
    	for j = 1:dim
    		if mod(i,2) == 0    			
    		M11(i,j) + M11(i+1,j) == M11(1,j);
    		M12(i,j) + M12(i+1,j) == M12(1,j);
			M21(i,j) + M21(i+1,j) == M21(1,j);
			M22(i,j) + M22(i+1,j) == M22(1,j);
    		end
    		if mod(j,2) == 0
    		M11(i,j) + M11(i,j+1) == M11(i,1);
    		M12(i,j) + M12(i,j+1) == M12(i,1);
    		M21(i,j) + M21(i,j+1) == M21(i,1);
    		M22(i,j) + M22(i,j+1) == M22(i,1);   		    		
	    	end
    	end
    end

    % Ensure that the matrix is PSD.
    M >= 0;
cvx_end
\end{verbatim}
\color{black}
\color{lightgray} 
\begin{verbatim}     
cvx_optval =

    0.75783
\end{verbatim}
\color{black}


%-------------------------------------------------------------------------------
\subsection{The non-signaling value for the CHSH extended nonlocal game} \label{code:ns-val-chsh-enlg}
%-------------------------------------------------------------------------------

\begin{verbatim}
n = 1;
dim = 2^n;

e0 = [1;0];         e1 = [0;1];
ep = [1;1]/sqrt(2); em = [1;-1]/sqrt(2);
eip = (e0 + 1j*e1)/sqrt(2); eim = (e0 - 1j*e1)/sqrt(2);

psi0_dm = e0*e0'; psi0_dmc = e1*e1';
psi1_dm = ep*ep'; psi1_dmc = em*em';
psi2_dm = eip*eip'; psi2_dmc = eim*eim';

P = zeros(2,2,2,2);
%P(:,:,1,1) = (psi0_dm)/2; P(:,:,1,2) = (psi0_dmc)/2;
%P(:,:,2,1) = (psi1_dm)/2; P(:,:,2,2) = (psi1_dmc)/2;

P = zeros(2,2,2,2,2,2);
P(:,:,1,1,1,1) = psi0_dm/2;
P(:,:,1,1,2,2) = psi0_dmc/2;

P(:,:,1,2,1,1) = psi0_dm/2;
P(:,:,1,2,2,2) = psi0_dmc/2;

P(:,:,2,1,1,1) = psi0_dm/2;
P(:,:,2,1,2,2) = psi0_dmc/2;

P(:,:,2,2,1,2) = psi1_dm/2;
P(:,:,2,2,2,1) = psi1_dmc/2;

cvx_begin sdp 
    %#ok<*VUNUS>    % suppress MATLAB warnings for equality checks in CVX
    %#ok<*EQEFF>    % suppress MATLAB warnings for inequality checks in CVX
     
    variable rho(dim,dim,dim,dim,dim,dim) semidefinite
    variable sig(dim,dim,dim,dim) hermitian
    variable xi(dim,dim,dim,dim) hermitian
    variable tau(dim,dim) hermitian 
        
    % construct objective function
    obj_fun = 0;
    for x = 1:dim
        for y = 1:dim
            for a = 1:dim
                for b = 1:dim
                   obj_fun = obj_fun + ip( P(:,:,x,y,a,b), rho(:,:,x,y,a,b) );
               end
           end
       end
    end
    
    maximize  obj_fun
    
    subject to 
    
    rho_b_sum = sum(rho,6);
    for x = 1:dim
        for y = 1:dim
            for a = 1:dim
                rho_b_sum(:,:,x,y,a) == sig(:,:,x,a);
            end
        end
    end
    
    rho_a_sum = sum(rho,5);
    for x = 1:dim
        for y = 1:dim
            for b = 1:dim
                rho_a_sum(:,:,x,y,b) == xi(:,:,y,b);
            end
        end
    end
    
    sig_a_sum = sum(sig,4);
    xi_b_sum = sum(xi,4);
    for x = 1:dim
        sig_a_sum(:,:,x) == tau;
    end
    for y = 1:dim
        xi_b_sum(:,:,y) == tau;
    end
    
    trace(tau) == 1; 
    tau >= 0;          
                        
cvx_end
cvx_optval
\end{verbatim}
\color{lightgray} 
\begin{verbatim}     
cvx_optval =

    0.75
\end{verbatim}
\color{black}

%-------------------------------------------------------------------------------
\subsection{Implementation of the see-saw method for computing lower bounds on the BB84 extended nonlocal game} \label{code:bb84-enlg-lower-bound}
%-------------------------------------------------------------------------------

\begin{verbatim}
e0 = [1;0];         e1 = [0;1];
ep = [1;1]/sqrt(2); em = [1;-1]/sqrt(2);

psi0_dm = e0*e0'; psi0_dmc = e1*e1';
psi1_dm = ep*ep'; psi1_dmc = em*em';

lvl = 1;
reps = 1;    
j_max = 4;

xdim = 2;
ydim = 2;

num_inputs = 2;
num_outputs = 2;

I = eye(xdim,ydim);

R = zeros(2,2,2,2,2,2);
R(:,:,1,1,1,1) = psi0_dm/2;
R(:,:,1,1,2,2) = psi0_dmc/2;
R(:,:,2,2,1,1) = psi1_dm/2;
R(:,:,2,2,2,2) = psi1_dmc/2; 

best = 0;
    
for k = 1:j_max
    k
    
    % Generate random bases from the orthogonal colums of randomly 
    % generated unitary matrices.     
    B = zeros(xdim,ydim,num_inputs,num_outputs);
    for y = 1:num_inputs
        U = RandomUnitary(num_outputs);
        for b = 1:num_outputs
            B(:,:,y,b) = U(:,b)*U(:,b)';
        end
    end  


    % Run the actual alternating projection algorithm between
    % the two SDPs. 
    it_diff = 1;
    prev_win = -1;
    while it_diff > 10^-6
        % Optimize over Alice's measurement operators while
        % fixing Bob's. If this is the first iteration, then the 
        % previously randomly generated operators in the outer loop are
        % Bob's. Otherwise, Bob's operators come from running the next
        % SDP.
        cvx_begin sdp quiet                
            variable rho(xdim^(2*reps),ydim^(2*reps),...
            num_inputs,num_outputs) hermitian
            variable tau(xdim^(2*reps),ydim^(2*reps)) hermitian

            win = 0;
            for x = 1:num_inputs
                for y = 1:num_inputs
                    for a = 1:num_outputs
                        for b = 1:num_outputs                               
                            win = win + ...                            
                           	trace( (kron(R(:,:,x,y,a,b), ...
                           	B(:,:,y,b)))' * rho(:,:,x,a) );                                
                        end
                    end
                end
            end
            
            maximize real(win)

            subject to 
                
                % Sum over "a" for all "x". 
                rho_a_sum = sum(rho,4);
                for x = 1:num_inputs
                    rho_a_sum(:,:,x) == tau;
                end
                                                                           
                % Enforce that tau is a density operator.
                trace(tau) == 1;
                tau >= 0;
                                   
                rho >= 0; 

        cvx_end            
        win = real(win);
                   
        % Now, optimize over Bob's measurement operators and fix 
        % Alice's operators as those coming from the previous SDP.
        cvx_begin sdp quiet
                        
            variable B(xdim,ydim,num_inputs,num_outputs) hermitian 
                                                         
            win = 0;
            for x = 1:num_inputs
                for y = 1:num_inputs
                    for a = 1:num_outputs
                        for b = 1:num_outputs                                                               
                            win = win + ...
                             trace( (kron(R(:,:,x,y,a,b), ...
                             B(:,:,y,b)))' * rho(:,:,x,a) );
                        end
                    end
                end
            end     
            
            maximize real(win)
            
            subject to 
                          
                % Bob's measurements operators must be PSD and sum to I
                B_b_sum = sum(B,4);
                for y = 1:num_inputs
                    B_b_sum(:,:,y) == I;
                end
                B >= 0;                                                           
                             
        cvx_end           
        win = real(win);
        
        it_diff = win - prev_win;
        prev_win = win;
     end
    
    % As the SDPs keep alternating, check if the winning probability
    % becomes any higher. If so, replace with new best.
    if best < win
        
        best = win;

        % take purification of tau
        pur = PartialTrace(tau,2);            

        A = zeros(xdim,ydim,num_inputs,num_outputs); 
        for x = 1:num_inputs
            for a = 1:num_outputs
               A(:,:,x,a) = pur^(-1/2) * PartialTrace(rho(:,:,x,a),2) * pur^(-1/2); 
            end
        end
         
        opt_strat_A = A;
        opt_strat_B = B;
    end

end;
 
best
\end{verbatim}
\color{lightgray} 
\begin{verbatim}     
best =

    0.8536
\end{verbatim}
\color{black}

%-------------------------------------------------------------------------------
\subsection{The BB84 monogamy game (Example \ref{ex:bb84-monogamy-game})} \label{code:bb84-game}
%-------------------------------------------------------------------------------

\begin{verbatim}
% Create the BB84 basis.
e0 = [1;0]; e1 = [0;1];
ep = [1;1]/sqrt(2); em = [1;-1]/sqrt(2); 

psi0 = e0*e0'; psi1 = e1*e1';
psip = ep*ep'; psim = em*em'; 

% Referee's first basis: {|0><0|, |1><1|}
R{1} = {psi0,psi1};

% Referee's second basis: {|+><+|, |-><-|}
R{2} = {psip,psim};

% BB84 game for a single repetition.
reps = 1; 

% Level of the extended QC hierarchy 
lvl = 1;

% Calculate the lower and upper bounds on the BB84 game:
%   cos^2(pi/8) \approx 0.8536
lb = MonogamyGameValueLB(R,reps,lvl)
ub = MonogamyGameValueUB(R,reps,lvl)
\end{verbatim}\color{lightgray} 
\begin{verbatim}     
lb =

    0.8535
ub =

    0.8535
\end{verbatim}
\color{black}

%-------------------------------------------------------------------------------
\subsection{A monogamy-of-entanglement game defined by mutually unbiased bases (Example \ref{ex:mub-4-3-monogamy-game})} \label{code:mub-4-3}
%-------------------------------------------------------------------------------

\begin{verbatim}
% Number of inputs and outputs
nin = 4;
nout = 3;

% Create the mutually unbiased bases consisting of 4-inputs and 3-outputs.
m = MUB(nout); 
R = {};
for i = 1:nin
    for j = 1:nout
        R{i}{j} = m{i}(:,j) * m{i}(:,j)';
    end
end

% Number of repetitions of the game. 
reps = 1; 

% Level of the extended QC hierarchy.
lvl = 1;

% Calculate the lower and upper bounds on the quantum value of
% the mutually unbiased basis game:
lb = MonogamyGameValueLB(R,reps,lvl)
ub = MonogamyGameValueUB(R,reps,lvl)
\end{verbatim}
\color{lightgray} 
\begin{verbatim}     
lb =

    0.6610
ub =

    0.6667
\end{verbatim}
\color{black}

%-------------------------------------------------------------------------------
\subsection{A counter-example to strong parallel repetition for monogamy-of-entanglement games with non-signaling provers (Proof of Theorem \ref{thm:no-spr-non-signaling})} \label{code:ns-counter-example-monogamy-game}
%-------------------------------------------------------------------------------

\begin{verbatim}
% Create the BB84 basis.
e0 = [1;0]; e1 = [0;1];
ep = [1;1]/sqrt(2); em = [1;-1]/sqrt(2); 

psi0 = e0*e0'; psi1 = e1*e1';
psip = ep*ep'; psim = em*em'; 

% Referee's first basis: {|0><0|, |1><1|}
R{1} = {psi0,psi1};

% Referee's second basis: {|+><+|, |-><-|}
R{2} = {psip,psim};

% BB84 game for a single repetition.
reps = 1; 

% Level of the extended QC hierarchy corresponds to non-signaling 
lvl = 0;

% Calculate the lower and upper bounds on the BB84 game:
rep_1_val = MonogamyGameValue(R,reps,lvl)

% BB84 game for a single repetition.
reps = 2; 

% Calculate the lower and upper bounds on the BB84 game:
rep_2_val = MonogamyGameValue(R,reps,lvl)
\end{verbatim}
\color{lightgray} 
\begin{verbatim}     
rep_1_val =

    0.8536

rep_2_val =

    0.7383
\end{verbatim}
\color{black}